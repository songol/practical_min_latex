\documentclass{resume}
\usepackage[left=0.75in,top=0.6in,right=0.75in,bottom=0.6in]{geometry}
\newcommand{\tab}[1]{\hspace{.2667\textwidth}\rlap{#1}}
\newcommand{\itab}[1]{\hspace{0em}\rlap{#1}}
\name{Vladimir Lenin}
\address{(+7)5555555 \\ gogocommunism@gmail.com}
\begin{document}


\begin{rSection}{Education}

{\bf Imperial Kazan University} \hfill {\em 1888 - 1889} 
\\ Junior Undergraduate \hfill { Overall GPA: /10}
\\ Department of Law 
{\bf Saint Petersburg Imperial University} \hfill {\em 1890 - 1891} 
\\ Junior Undergraduate \hfill { Overall GPA: /10}
\\ Department of Law  

\end{rSection}

\begin{rSection}{Technical Strengths}

\begin{tabular}{ @{} >{\bfseries}l @{\hspace{6ex}} l }
Master of Marx's theory
\end{tabular}

\end{rSection}


\begin{rSection}{Experience}

\begin{rSubsection}{October Revolution: 1917}{november 1917}{CEO}{}
\item Attended a meeting of the Bolshevik Central Committee on 10 October, where he again argued that the party should lead an armed insurrection to topple the Provisional Government
\end{rSubsection}

\begin{rSubsection}{Organising the Soviet government}{1917–1918
}{Succeeded by	Alexei Rykov
Chairman of the Council of People's Commissars of the Russian SFSR}{}
\end{rSubsection}

\end{rSection}


\begin{rSection}{Achievements} \itemsep -2pt
\item Order of the Labor of the Khorezm People's Soviet Republic 1922 y
\end{rSection}

\begin{rSection}{Favourite math formula} \itemsep -2pt
$$\sup_{\theta\in\Theta}||\frac{1}{n}\sum\limits_{i=1}^n f(X_i, \theta)-\math{E}[f(X,\theta)]|| \stackrel{a.s.}\longrightarrow 0$$
\end{rSection}


\end{document}
